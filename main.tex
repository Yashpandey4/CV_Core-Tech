\documentclass{article}

\usepackage{verbatim}
\usepackage{hyperref}
\hypersetup{
  colorlinks=true,
  urlcolor=black,      % color of file links
  }
%\usepackage[scaled]{helvet}
%\renewcommand*\familydefault{\sfdefault}

\newenvironment{longversion}{}{} % use this to show longversion
%\newenvironment{longversion}{\comment}{\endcomment} % use this to hide  longversion
\newenvironment{shortversion}{}{} % use this to show shortversion
%\newenvironment{shortversion}{\comment}{\endcomment} % use this to hide shortversion

\usepackage{changepage}
\usepackage{tabularx}
\usepackage{setspace}
\usepackage{url}
\usepackage{sectsty}
\usepackage[letterpaper,margin=0.45in]{geometry}
\pagestyle{empty}

\newenvironment{absolutelynopagebreak}
  {\par\nobreak\vfil\penalty0\vfilneg
   \vtop\bgroup}
  {\par\xdef\tpd{\the\prevdepth}\egroup
   \prevdepth=\tpd}

% indentsection style, used for sections that aren't already in lists
% that need indentation to the level of all text in the document
\newenvironment{indentsection}[1]%
{\begin{list}{}%
	{\setlength{\leftmargin}{#1}}%
	\item[]%
}
{\end{list}}

% opposite of above; bump a section back toward the left margin
\newenvironment{unindentsection}[1]%
{\begin{list}{}%
	{\setlength{\leftmargin}{-0.5#1}}%
	\item[]%
}
{\end{list}}

% format two pieces of text, one left aligned and one right aligned
\newcommand{\headerrow}[2]
{\begin{tabular*}{\linewidth}{l@{\extracolsep{\fill}}r}
	#1 &
	#2 \\
\end{tabular*}}

% make "C++" look pretty when used in text by touching up the plus signs
\newcommand{\CPP}
{C\nolinebreak[4]\hspace{-.05em}\raisebox{.22ex}{\footnotesize\bf ++}}

%edit the section font and style
\sectionfont{\normalfont\sectionrule{0pt}{0pt}{-4pt}{1pt}}

%make all sections cap and first letter capital
\newcommand{\tmpsection}[1]{}
\let\tmpsection=\section
\renewcommand{\section}[1]{\tmpsection*{\textsc{#1}}}

%set the line spacing
\setstretch{1.0}


\begin{document}
%TITLE


\begin{center}
 {\LARGE \textsc{Pratyush Pandey} }\\ 

\end{center}

	
\begin{minipage}{.45\linewidth}
\begin{flushleft}                           
\textbf{Sophomore - Electrical Engineering} \\
Indian Institute of Technology Delhi \\
+91-9820807802 
\end{flushleft} 
\end{minipage}
\hfill
\begin{minipage}{.45\linewidth}
\begin{flushright}                                      
yashpandey4@gmail.com  \\
   \href{https://github.com/Yashpandey4}{github.com/Yashpandey4} \\
  \href{https://www.linkedin.com/in/pratyush-pandey-b99b5b143/}{linkedin.com/in/pratyush-pandey-b99b5b143/}\\ 
\end{flushright} 
\end{minipage}

\section{Academic Details}

\begin{center}
\begin{tabular}{ |c | c | c | c |}
\hline
Year & Degree & Institute & CGPA/Percentage \\ 
\hline
2017-2021 & B.Tech in Electrical & Indian Institute of Technology (IIT) & 8.632 \\ 
\textbf{(Current)} &  and Electronics Engineering & Delhi & \\
\hline


2017 & Class XII, HSC & \begin{tabular}{@{}c@{}}Ratanbai Walbai Junior College of Science, \\Mulund \end{tabular} & 90\% \\ 

\hline
2015 & Class X, ICSE & \begin{tabular}{@{}c@{}}Smt.Sulochanadevi Singhania School, \\Thane \end{tabular} & 98.2\% \\  \hline
\end{tabular}
\end{center}

\section{Scholastic Achievements}
\begin{itemize}
    \setlength\itemsep{0.0em}
    
    \item \textbf{Cargill Global Scholarship 2019:} \textbf{1 in 10} Indian awardees among 30K applicants from 6 countries for this leadership development cum scholarship program, which connects you to a global network of Cargill scholars.
    
    \item \textbf{Publication:} Co-authored \textit{MergeConnects: Generalized regularisation of Deep Neural Networks}, under review in IJCAI 2020 (Macao, china)
 
    \item \textbf{IITD Semester merit Award}: Awarded after securing the \textbf{Institute highest 10 CGPA} and finishing in \textbf{top 7\%} in the fall semester, '17-'18.
    
    % \item Secured a place \textbf{among Top 0.05\%} of the students in the IIT Joint Entrance Exam(IIT-JEE) 2017 out of  \textbf{more than 11,00,000} students who appeared for the examination.
    
   \item \textbf{Change of Department}: \textbf{1 in 6} students (among 950) in IITD selected to change their majors to Electrical Engineering based on CGPA and co-curricular criteria. 
   
    \item \textbf{KVPY Scholar 2017}: Granted the `Kishore Vaigyanik Protsahan Yojana' award by Dept of Science and Technology, Govt of India.

    \item \textbf{Siemens Scholarship '15}: Awarded for academic excellence and securing a place in \textbf{top 10 ICSE scorers} in the country.
    
%   \item \textbf{OnePlus Student Ambassador '18}: \textbf{1 in 11 ambassadors selected from a pool of 10K} applicants all over India. Learnt essential leadership, event management and marketing skills. Received LoR for work done.
   
    % \textbf{Sulonia Honour Society}: Member for 5 years in a row (2009 - 2014) due to academic, extra-curricular and all round achievements, part of school senate.
    
    %\item Special mention in \textbf{Microsoft's Code.Fun.Do} campus wide Hackathon in 2018.
    
\end{itemize}

\section{Major Internships}
% \begin{spacing}{1}
\begin{list} {\labelitemi}{\leftmargin=0em}
\setlength{\leftmargin}{0pt}
%   As a Data Scientist (and Software Engineer) at Bending Spoons I try to understand the behavior and measure the value of millions of users of our apps:
% - I developed a statistical model to predict future number of purchases given a small observation period as my master thesis in Bending Spoons. This is widely used to predict future incomes and to get quicker results for a/b testing.
% - I currently give support to the marketing team building software tools to compare costs of acquisition and expected revenues from users (based on the above cited model). This makes them able to make informed decisions and improve our investments.
      \item[]
    \headerrow {\textbf{Bending Spoons - Data Science and SDE Intern}  \textit{(Ongoing Internship)}}{\textbf{Milan, Italy},  Nov 2019 - \textit{Present}}
    \begin{itemize}
    \setlength\itemsep{0.0em}
        \item Currently working on internal tools for marketing optimization and providing support to the Marketing Technology team. Working with Python, PostgreSQL, MongoDB, Google BigQuery, Docker, Google Cloud Platform.
        \item Building software tools to compare costs of acquisition and expected revenues from users, based on a statistical model to predict future purchasing behaviour of user.
    \end{itemize}
    
    \item[]
    \headerrow {\textbf{Regularisation of Deep Neural Networks, SUTD, Singapore}}{Prof. Ernest Chong (Cornell), May - July 2019}
    Worked as a research assistant in Singapore University of Technology and Design (SUTD), in the ISTD pillar in Dr Ernest Chong's research group.
    \begin{itemize}
    \setlength\itemsep{0.0em}
        \item Generalised the Dropout Regularisation method of Deep Neural Networks to a novel two model training approach called "MergeConnect"; enhanced generalisation over unseen data by 2\% over dropouts/connect.
        \item Outperformed dropout on state of art ResNet32, wide ResNet \& ResNeXT network architectures using benchmarking datasets SVHN, CIFAR-10 and CIFAR-100.
        \item Tested method on ResNeXt with skip-connection, data-augmentation \& momentum to achieve \textbf{96.4\% accuracy} on CIFAR-10 dataset, improving error rate over Dropouts by 0.8\%.
    \end{itemize}
    
    \item[]
    \headerrow {\textbf{Siemens Ltd. - "FutureLand"}}{Mr.Ravi Subramanium (IIT-B, IIM-A), May - July 2018}
    Worked in Energy Management - Strategy division in Siemens, Mumbai. Received a \textbf{Letter of Recommendation} from the supervisor and head of division, Mr.Ravi Subramanium.
    \begin{itemize}
    \setlength\itemsep{0.0em}
        \item Created ML classification model for predicting time of breakdown of Siemens motors, based on its operational parameters (vibrations, temperature, rotation speed, etc) at the time of production. Detailed Report of the work is present \href{https://csciitd-my.sharepoint.com/:b:/g/personal/ee1170938_csciitd_onmicrosoft_com/EasUavDrTmdFrvuYc0rOissBYh8X1WvKKr_OTclVrCd42w?e=wCPjfb}{here}.
        \item Used Naive-Bayes \& Random Forest algorithms for classification to achieve 90.2\% accuracy on the training data, fetched from Mindsphere - Siemens' open course IoT cloud platform, linked to all motors and production systems.
        \item Held Training sessions on the FutureLand focus topics \href{https://www.siemens.com/global/en/home/company/fairs-events/futureland.html}{(link)} to sensitize 400 Workers and Executives from various departments.
    \end{itemize}
    

    % \item[]
    % \headerrow {\textbf{Data Structures Intern}}{Prof. Maya Ramanath, May-July 2018}
    % Work appreciated by Professor, Received \textbf{Letter of Recommendation} and \textbf{Semester Credits} for the same. A link to the project is \href{https://github.com/Yashpandey4/BTree_Vis}{here}
    % \begin{itemize}
    % \setlength\itemsep{0.0em}
    %     \item Built software for generating interactive graphical simulations of the BTree, B*Tree, and B+Tree data structures used in Database Management Systems.
    %     \item Employed addition, deletion, search, and traverse operations, added support for tree of any degree \textgreater=2.
    %     \item Used the external JGraphX,  library to render graphics and animate the above operations.
    %     \item Experimented with JavaFX to render animations, Explored it's application in areas like debugging complex database storage systems by visualization.
    % \end{itemize}
    
    %  \item[]
    % \headerrow {\textbf{OnePlus Ltd., Bangalore}}{September 2018 - March 2019}
    % Among 11 students selected from across the country by OnePlus for their first edition of Campus Connect Programme \href{https://www.oneplus.in/campus}{(Details here)}.
    % \begin{itemize}
    % \setlength\itemsep{0.0em}
    %     \item Ideated on and executed innovative campaign plans for OnePlus Ltd.
    %     \item Worked on live marketing projects for the brand
    %     \item Strategised and drove Word of Mouth in campus events and activities.
    %     \item Formed a network of like-minded students across India to serve as OnePlus community specialists, brand evangelists and technology experts in their respective institutions.
    % \end{itemize}

\end{list}
% \end{spacing}

% \begin{absolutelynopagebreak}
\begin{longversion}
\section{Major Projects}
\begin{list} {\labelitemi}{\leftmargin=0em}
\setlength{\leftmargin}{0pt}
% \setlength\itemsep{5em}
%\begin{itemize}

 \item[]
\headerrow { \textbf{Cannon playing AI Bot}} {Prof. Mausam, IIT Delhi , August 2019 - Nov 2019}
 \begin{itemize} \item[]
Cannon is a two-player abstract strategy game whose rules can be found \href{https://nestorgames.com/rulebooks/CANNON_EN.pdf}{here}. Working in a team of two, we developed a weighted utility function for IDDFS Search (upto depth 8) using 11 game features. Search was done using Minimax strategy, improved efficiency was obtained with the aid of alpha-beta pruning (with moves ordered in decreasing order of likelihood for better pruning), quiescent search, transposition table and tabu list. TD Learning was implemented to learn the weights of the features used in the utility function. These functions were implemented as efficiently as possible as bot was allocated no more than 200 seconds to make all moves and win.
 \end{itemize}

 \item[]
\headerrow { \textbf{Visualising activations of DNNs \& Sparse Autoencoders}} {Prof. Sumeet Agarwal, IIT Delhi , Oct 2019 - Nov 2019}
 \begin{itemize} \item[]
Explored the high level or hidden representation learnt by Deep Neural Networks (DNNs) when trained on MNIST dataset, and compared results with other data dimensionality reduction techniques (PCA). Implemented standard backpropogation fully connected net, Convolutional nets and K-Sparse Auto-encoders from scratch, along with dropout regularisation techniques. Plotted the activations of various layers using Matplotlib to visualise the final learnt representation. Went on to implement these networks using Tensorflow and keras, along with the use of Nesterov Momentum, Boosting and Adam Optimiser for training, to achieve highest final accuracy of 99.3\%. The full report can be found \href{https://csciitd-my.sharepoint.com/:w:/g/personal/ee1170938_csciitd_onmicrosoft_com/EVa7dWSz8t1Dlt3IZ5EXa_QB0ppdpWLaf5AHSk3PCPcv0g?e=wS6EWa}{here.}
 \end{itemize}
 
 \item[]
\headerrow { \textbf{Muticlass SVM for MNIST classification}} {Prof. Sumeet Agarwal, IIT Delhi , September 2019 - Oct 2019}
 \begin{itemize} \item[]
Implement a one-vs-all SVM classifier for classification of the full MNIST dataset. Implemented the KKT conditions and solved the primal problem via convex optimisation using the CVXOPT library. Trained the classifier using batch gradient descent, 10-fold cross validation, and exponential decay learning rate. Tested the architecture using RBF, polynomial and guassian kernels. Achieved 85.6\% accuracy on the test data. The full report can be found \href{https://csciitd-my.sharepoint.com/:b:/g/personal/ee1170938_csciitd_onmicrosoft_com/ER9gs-cvP5tBtehLDkQoJzsBKB3wo4UxdHrTofb7r1Eo5w?e=cjrFgh}{here.}
 \end{itemize}

 \item[]
\headerrow { \textbf{Graph Subset Mapping using SAT Solvers}} {Prof. Mausam, IIT Delhi , September 2019 - Oct 2019}
 \begin{itemize} \item[]
Formulated the NP-Hard Graph subset mapping problem as a SAT problem and solved it using MiniSAT solver. Used heuristics to solve the problem efficiently by reduce variables and clauses, causing a 10-fold reduction in a random directed graph. To make encoding generation time competent with sat solving time, several C code optimisations were made - including Buffered inputs, cache-friendly loops and DP for conversion of Strings to integers. Other optimisations included removal of isolated nodes using constraints on in and out-degrees of graph nodes to eliminate clauses.
 \end{itemize}

\item[]
\headerrow { \textbf{Blur reduction in imaging fast moving targets}} {Prof. Vikram Gadre, IIT Bombay, July 2018 - May 2019}
 \begin{itemize} \item[]
Currently working in Prof Vikram Gadre's research group in IIT Bombay to improve the image quality obtained from Synthetic Aperture Radars (SARs). Improved the processing of linear, quadratic and cubic chirp signals from SARs using GTFT (Generalised Time fourier transform) to obtain focused SAR image of ground moving targets using AGFS (Adaptive Generalised Frequency Spectrogram) based parameter estimation. The project has funding from DRDO, Bangalore and has applications in the military. 
% Paper expected to be published in 
 \end{itemize}
 
 \item[]
\headerrow { \textbf{Visualisation of Data Structures}} {Prof. Maya Ramanath, IIT Delhi , May 2018 - July 2018}
 \begin{itemize} \item[]
Built software for generating interactive graphical simulations of the BTree, B*Tree, and B+Tree data structures used in Database Management Systems. The link to the project is \href{https://github.com/Yashpandey4/BTree_Vis}{here}. Used the external JGraphX,  library to render graphics and animate the above operations. Experimented with JavaFX to render animations.
 \end{itemize}

\item[]
\headerrow { \textbf{Tumour Detection in Brain Using Pre-Processing of MRI images}} {Prof. SD Joshi, IITD, July 2018 - Dec 2018}
 \begin{itemize} \item[]
Performed image enhancement and noise reduction
techniques on MRI images, then applied morphological operations to detect the tumor in the image. In the end the tumour was mapped onto
the original gray scale image with 255 intensity to make
visible the tumour in the image. The algorithm used had two stages, first is pre-processing by converting to gray-scale image then applying high pass and median filters for image enhancement and noise reduction, and after that computing the threshhold and watershed segmentation, and finally performing morphological operations. A detailed project report can be found  \href{https://drive.google.com/file/d/1SlMhfu_FqdmNmo-ZZoaVuqLGvVw4tHOT/view?usp=sharing}{here}.
 \end{itemize}
 
%  \item[]
% \headerrow { \textbf{Finding dominant strategy in two person stochastic games}} {Dr. Manoj Gopal, IIT Bombay , Nov 2018 - Apr 2019}
%  \begin{itemize} \item[]
% Collaborated with students from Indian Statistical Institute (ISI) under the supervision of Prof Manoj Gopalakrishnan to solve the problem of two person stochastic games with a notion of power. Made use of Markov decision processes and Bellman Ford Algorithm to compute the best mixed strategy response to the opponent's mixed strategy. Computed payoff matrix in each step of the stochastic play and studied the effects of the two players having different 'power'.
%  \end{itemize}
 
%  \headerrow { \textbf{Tracking Motion and Behavioural Patterns of Fishes}} {Prof. Aditi, Aug 2018 - Dec 2018}
%  \begin{itemize} \item[]
%  Worked with the Samsung Innovations lab in IIT Delhi to study a correlation between the random motion of fishes during different times of day and guess their behavioral patterns. Different species of fishes are being studied in tanks of different sizes and depth. The data is recorded via Kinect Studio and Processed and Analysed via MATLAB to find behavioral patterns and their correlations with the surrounding condition, in hope to obtain insight over thinking process of fishes. Phase 2 of the project to begin in October, 2018.
%  \end{itemize}
 
%   \item[]
% \headerrow { \textbf{Optimum Applications of Data Structures}} {Prof. Amitabha Bagchi, IIT Delhi, July - Dec 2018}
%  \begin{itemize} \item[]
%  Over a course of 6 months, modeled and implemented various real life objects like a Mobile Phone tracking and routing structure, a Social Media Platform and a Search Engine using Data structures in java for quick and optimal implementation using various data structures and algorithms. Received Semester credits for the same. The code is up on my Github.
%  \end{itemize}

%  \item[]
% \headerrow { \textbf{Cell Phone Detector Circuit}} {Prof. Shouribrata Chatterjee, IIT Delhi, Jan - Mar 2018}
%  \begin{itemize} \item[]
%  Designed and developed a circuit which can detect Cell-Phones operational in a 10 meter radius by amplifying the specific long range signals emitted by them using high frequency chokes. Incorporated Schottky diode to rectify low frequency AC signal and ceramic capacitor to filter out AC ripples. BJT was used to amplify the signal and in conjunction with a Comparator to um-ambiguously detect a cell-phone signal.
%  \end{itemize}
 
% \headerrow { \textbf{Affordable Energy Generator}} {Prof. PVM Rao, July - Dec 2017}
%  \begin{itemize} \item[]
%  Designed and assembled a Cheap, Clean and Efficient electricity generator from scratch using a dynamo, a bicycle wheel and a step up transformer to store this energy. The design of the product was not patented and it can be manufactured for less than 100 rupees to provide cheap power to those living in rural areas in India. A presentation to the project is \href{https://csciitd-my.sharepoint.com/:p:/g/personal/ee1170938_csciitd_onmicrosoft_com/EaSyGhWUxVlJg2bV1XZGJvAB0vl_h_SEt4NfQzvOdwgxYQ?e=KFC30i}{here}.
%  \end{itemize}

% \headerrow {\textbf{Content Intern, Indian Road Safety Campaign}} {NEN201 Internship, March-April, 2018}
% \begin{itemize} \item[]
% NEN201 is a Professional Ethics and Social Responsibility Course, which requires us to volunteer in an Non-Government Organisation. Indian Road Safety Campaign \href{https://www.road-safety.co.in/}{(link)} is the largest youth led organisation in India. As a part of my intern we visited various part of Delhi and Mumbai to conduct various events and spread awareness about safety on road, all in collaboration with the Delhi Traffic Police. Being the content writer, I managed their official Social Media Handles, blogs and Newsletters to spread awareness about the same. Direct outreach exceeds \textbf{1 Million People} who are influenced by awareness campaigns every year.
% \end{itemize}

% \item[]
% \headerrow {\textbf{Apocalypse}} {Independent Project, May - Dec 2017}
% \begin{itemize} \item[]
% Created A Windows/Mac/Linux Game made for PC on Unity. It is a survival game coded in C\#. A link to project is \href{https://github.com/Yashpandey4/Apocalypse-the-game}{here}.
% \end{itemize}

\end{list}
%\end{itemize}

% \pagebreak

\begin{longversion}
\section{Technical Skills}\begin{itemize}
\item \textbf{Programming Languages:}  \CPP, Python, Java, C, Swift, MATLAB, OCaml, C\#, R, Ruby
\item \textbf{Frameworks:} TensorFlow, PyTorch, Keras, Docker, Django, Flask, PostgreSQL, MongoDB, ReactJS, Git, $\LaTeX$
\item \textbf{Softwares:} Android Studio, Xcode, Wolfram Mathematica, Unity 3D, Arduino, Autodesk, Solidworks, Microsoft Office

\end{itemize}

\end{longversion}

% \textit{*All projects are available in my github profile}
\end{longversion}

\begin{longversion}
\section{Relevant Courses}
\begin{itemize}
\setlength\itemsep{-1em}

\item \textbf{Computer Science:} \hfill \textit{(*Courses currently pursuing)}\\
Artificial Intelligence; Machine Learning and Intelligence; Database Management Systems; Analysis and Design of Algorithms; Data Structures and Algorithms; Computer Architecture\\

\item \textbf{Electrical:}
Physical Electronics; Signals and Systems; Electromagnetism; Circuit Theory; Digital Electronics; Engineering Electromagnetics; Control Engineering; Analog Electronic Circuits, Communications Engineering, Power Electronics\\ 

\item \textbf{Mathematics:} Calculus, Linear Algebra, Differential Equations, Probability and Stochastic Processes\\

\item \textbf{Online:}
Deep Learning (\href{http://www.fast.ai/}{Fast.ai}, Coursera), Intro to CS (CS50, Harvard), Machine Learning (Coursera)

\end{itemize}
% \textit{*Courses currently pursuing}
\end{longversion}



\section{Extra Curricular Activities}
%\setlength{\leftmargin}{0pt}

\begin{itemize}
    \setlength\itemsep{0em}
    
    \item \underline{\smash{\textbf{Chief Editor}, and \textbf{(former) Journalist} at Board of Student Publications, IIT Delhi.}} \href{http://bsp.iitd.ac.in/}{(link)}\\
    - Heading the creative segment of BSP; \textbf{Leading a team of 30+ Student Journalists}; Responsible for online outreach \\
    - \textbf{Coordinator, Publicity, Literati '19:} IITD's annual Literary fest; Edited and headed Inception and Inquirer: our annual creative and journalistic magazines
    
    \item \underline{\smash{\textbf{Representative}: Debating Club}} - \textit{1st International Debating A-level in IITD}, multiple podium finishes \\
    - Executed 10+ events with participation of 500+ students from over India; conducted year-round workshops for freshers.\\
    \underline{International level}: Breaking Adjudicator, SMU Hammers '19 (Singapore) \& Malaysia Pre-Asians '19 (Kuala Lumpur).\\
    \underline{National level}: Breaking Adjudicator, IITPD '19 (Delhi); Contingent Best speaker, PEC Trivium'19 (Chandigarh).
    
    \item \textbf{Social work:} Youngest core-team member, Indian Road Safety Campaign (IRSC); Volunteer for NSS and BloodConnect\\
    \underline{\smash{IRSC:}} Handled social media handles and awareness campaigns which gathered 40k followers; Wrote brochures on road safety distributed in 20+ schools across Delhi NCR; helped conduct first responders workshop in AIIMS Delhi \\
    - Pitched traffic outflow optimisation strategies to MoRTH after surveying 50+ busy and congested roads in Delhi NCR\\
    \underline{\smash{Raymond Rehabilitation Center:}} Mentored 80+ underprivileged children in class 8 Science, Math \& English for 6 months

    \item \textbf{Digital Marketing:} Ideated \& led various marketing campaigns for OnePlus, Alibaba, Tryst, Pravega, Literati, IITPD\\
    \underline{\smash{OnePlus Student Ambassador '18:}} \textbf{1 in 10 students} selected in a pool of 10K; received \textbf{Letter of recommendation}\\
    - Provided logistic \& analytical support during \textbf{OnePlus 6T Launch Event} (Delhi) \& \textbf{6T McLaren Launch} (Mumbai)
    
    \item \textbf{Cultural Activities:} \underline{\smash{IITD:}} Member of Lit Club ($1^{st}$ in 3 Muses, $2^{nd}$ in Potpourri), Quizzing Club ($1^{st}$ in Freshers quiz)\\
\underline{\smash{Hindustani classical music:}} 6+ years of training; certified Visharad by Sangeet Kala Kendra(Bengal) and Gandharva(Pune)\\
\underline{\smash{Taekwondo:}} Brown Belt by World Karate Org.(WFSKO, Korea), recognised by Govt of India and Indian Olympic Assoc.

\item \textbf{Content Writing:} 5+years of content writing(30+articles) \& editing (5+magazines) experience for NGOs and Start-Up\\
- Wrote \textbf{3 featured articles} for Titanic App; Wrote poetry for BSP's Instagram and magazine; runs a personal blog\\
- Scriptwriter for play which finished \textbf{$2^{nd}$ in Mumbai} (Kasber's Live Wire'14); Editor for Hostel and School Magazines
    
    \item \textbf{Former Executive} at Office of Career Services (Previously Training and Placement Cell), IIT Delhi \href{https://tnp.iitd.ac.in/newtnp/}{(link)}
     
\end{itemize}

% Debating Rep Stuff
% Successfully coordinated and organised 2 National-level, 1 International-level and 7 institute-level events throughout the year. 
% ●Encouraged debating in the hostel by increasing enthusiasm through various internal workshops and exuberant activities.
% ●Improved commitment and punctuality of the organisation through ensuring attendance in regular short meetings.
% ●Increased participation by more than 300% by the end of my tenure.




% \end{absolutelynopagebreak}


\end{document}

